\documentclass[12pt,a4paper]{article}

\usepackage[utf8]{inputenc}
\usepackage[czech]{babel}
\usepackage{graphicx}
\usepackage{amsmath}
\usepackage{amssymb}
\usepackage{hyperref}
\usepackage{setspace}
\usepackage{enumitem}

\setlength{\parskip}{0.9em}
\setlength{\parindent}{0pt}

\title{Umělá inteligence - semestrální projekt \\[0.5em]
\large Analýza dat a aplikace vybrané metody strojového učení}
\author{
    Ondřej Chalupa \\ \texttt{chaluon3@student.cvut.cz}
    \and
    Matěj Hrdlička \\ \texttt{hrdlima8@student.cvut.cz}
}
\date{\today}

\begin{document}
\maketitle

\section*{Pokyny k práci}
Tento dokument slouží jako závazná šablona reportu. Vyplňte jednotlivé sekce v uvedeném rozsahu.  \\
Práce se řeší \emph{ve dvojicích}.

Datové sady musí pocházet z \emph{reálných nebo otevřených zdrojů} (ne knihovní benchmarky jako \texttt{Iris}, \texttt{MNIST}, \texttt{Breast Cancer}, \texttt{Wine}, apod.).  \\

Je povoleno využívat nástroje generativní AI (např. ChatGPT, Copilot). V závěru práce musí být uvedeno samostatné prohlášení popisující, k čemu byla AI využita (pouze pokud byla skutečně využita) (např. konzultace postupu, generování části kódu, kontrola textu). Využití AI však nesmí nahrazovat vlastní porozumění — každý student musí být schopen při ústním zkoušení detailně vysvětlit použitou metodu, její parametry, datovou přípravu, kód i interpretaci výsledků. Nesplnění této podmínky může ovlivnit hodnocení práce i ústní část zkoušky.

Celkový doporučený rozsah textu je \emph{8–12 stran bez příloh}.  \\

% --------------------------------------------------------------------

\section{Možné otevřené zdroje dat}
\begin{itemize}[leftmargin=*]
    \item \textbf{Kaggle Datasets:} \url{https://www.kaggle.com/datasets}
    \item \textbf{Google Dataset Search:} \url{https://datasetsearch.research.google.com}
    \item \textbf{UCI Machine Learning Repository (vyjma známých benchmarků):}  \\
          \url{https://archive.ics.uci.edu}
    \item \textbf{OpenML (vyhýbat se klasickým učebnicovým datasetům):}  \\
          \url{https://www.openml.org}
    \item \textbf{Data.gov (USA):} \url{https://www.data.gov}
    \item \textbf{EU Open Data Portal:} \url{https://data.europa.eu}
    \item \textbf{České otevřené datové portály:}  \\
    \begin{itemize}
        \item \url{https://data.gov.cz}  
        \item \url{https://opendata.praha.eu}  
        \item \url{https://opendata.brno.cz}
    \end{itemize}
    \item \textbf{OpenWeather, NOAA a další meteorologická data}
    \item \textbf{GitHub repozitáře s raw daty} (např. data ze senzorů, logů apod.)
    \item \textbf{Data z vlastního okolí} (pokud nejsou osobní – např. logy z chytré domácnosti, sportovní senzorika apod.)
\end{itemize}

% --------------------------------------------------------------------

\section{Úvod}

\begin{itemize}
    \item \textbf{Stručné představení vybrané metody a problému}
    
    Tato práce se zabývá aplikací strojového učení, konkrétně metody vícevrstvého perceptronu, na oblast analýzy dopravních přestupků v hlavním městě Praze. Data o dopravních přestupcích nám přišla zajímavá. Zjistili jsme, že záznamy o dopravních přestupcích v Praze přidává buď Policie České republiky, nebo Městská policie Praha. Chtěli jsme zjistit, jestli je možné na základě toho, že víme, kdy, kde a jaký přestupek byl spáchán, určit, zda přestupek řešila Policie České republiky, nebo Městská policie Praha.

    \item \textbf{Odůvodnění výběru dat a jejich relevance}
    
    Data jsme našli na portálu Open Data hlavního města Prahy. Na základě toho, že nám data o dopravních přestupcích přišla zajímavá, jsme si vymysleli problém, který s nimi můžeme vyřešit.

    \item \textbf{Struktura práce}
    
    Práce se skládá z několika souborů. První soubor \texttt{data\_process.py} obsahuje kód, který připraví data pro trénování nebo pro použití modelu na predikci. Soubor \texttt{train.py} obsahuje kód pro vytrénování modelu na datech a \texttt{predict.py} použije model pro predikci a určí jeho přesnost.
\end{itemize}

% --------------------------------------------------------------------

\section{Popis datové sady}

% Hlavní seznam dle struktury na obrázku
\begin{itemize}[leftmargin=*]

    % 1. BOD: PŮVOD DAT
    \item \textbf{Původ dat (zdroj a odkazy)}
    
    Data pocházejí z~portálu Open Data hlavního města Prahy a jsou vedena pod názvem \emph{Dopravní přestupky MHMP}. Pro trénování modelu byla zvolena data z~roku 2023, zatímco pro validaci a testování byly využity sady z~let 2022 a 2024.
    
    \noindent \textbf{Odkazy na zdroje:}
    \begin{itemize}
        \item Rok 2022: \href{https://opendata.praha.eu/dataset/dopravni-prestupky-mhmp-2022}{Dopravní přestupky MHMP 2022}
        \item Rok 2023: \href{https://opendata.praha.eu/dataset/dopravni-prestupky-mhmp-2023}{Dopravní přestupky MHMP 2023}
        \item Rok 2024: \href{https://opendata.praha.eu/dataset/dopravni-prestupky-mhmp-2024}{Dopravní přestupky MHMP 2024}
    \end{itemize}

    % 2. BOD: ATRIBUTY, VELIKOST, ČAS
    \item \textbf{Typy atributů, velikost a časové pokrytí}
    
    Datasety pokrývají období od ledna 2022 do června 2024. Sady z~let 2022 a 2023 obsahují záznamy za celých 12 měsíců, dataset z~roku 2024 obsahuje data pouze za první polovinu roku. Celkový přehled uvádí Tabulka~\ref{tab:pocty_zaznamu}.
    
    \begin{table}[h!]
        \centering
        \caption{Přehled počtu záznamů v~jednotlivých letech}
        \label{tab:pocty_zaznamu}
        \begin{tabular}{l c r}
            \hline
            \textbf{Dataset} & \textbf{Časové pokrytí} & \textbf{Počet záznamů} \\
            \hline
            MHMP 2022 & Leden -- Prosinec & 846\,832 \\
            MHMP 2023 & Leden -- Prosinec & 946\,701 \\
            MHMP 2024 & Leden -- Červen & 372\,084 \\
            \hline
            \textbf{Celkem} & & \textbf{2\,165\,617} \\
            \hline
        \end{tabular}
    \end{table}

    Každý záznam reprezentuje jeden evidovaný přestupek. Mezi klíčové atributy patří:
    \begin{description}
        \item[Časové údaje:] Datum (\emph{2022-04-15}) a Čas (\emph{09:14}).
        \item[Lokalizace:] Místo (\emph{Lohniského 4}) a Čtvrť (\emph{Praha 5}).
        \item[Subjekty:] Oznamovatel (\emph{MPP/PČR}), Státní příslušnost (\emph{CZ}), Značka vozidla.
        \item[Právní kvalifikace:] Porušený zákon (např. \emph{§ 125d zák. č. 361/2000 Sb.}).
    \end{description}

    % 3. BOD: EXPLORATORNÍ ANALÝZA
    \item \textbf{Exploratorní analýza: statistiky, distribuce a grafy}
    
    Analýza provedená na datech z~roku 2023 odhalila následující charakteristiky:
    \begin{itemize}
        \item \textbf{Distribuční poměr oznamovatelů:} Data vykazují silnou převahu záznamů od Městské policie (MPP), která tvoří přibližně 91,6\,\% datasetu (866\,709 záznamů). Policie ČR (PČR) se podílí zbylými 8,4\,\%.
        
        \item \textbf{Geografické rozložení:} Přestupky jsou nejčastěji evidovány v~obvodech \textbf{Praha 4} (207\,400 záznamů) a \textbf{Praha 6} (167\,246 záznamů). Centrum města (Praha 1) je v~četnosti až čtvrté.
        
        \item \textbf{Typy přestupků:} Nejčastějším porušením je §~125c odst.~1 písm.~k) (nedovolené zastavení/stání). Významnou část tvoří i překročení rychlosti do 20 km/h v~obci.
        
        \item \textbf{Struktura chybějících dat:} Atributy jako Datum či Oznamovatel jsou vyplněny vždy. Nejvíce chybějících hodnot vykazuje:
        \begin{itemize}
            \item tovární značka vozidla - 12.4\%
            \item čas - 4.6\%
            \item státní příslušnost - 	2.4\%
            
        \end{itemize}
 
    \end{itemize}

    % 4. BOD: IDENTIFIKACE PROBLÉMŮ
    \item \textbf{Identifikace problémů datasetu (nevyváženost, šum, chybějící hodnoty)}
    
    Během přípravy dat byly identifikovány tyto problémy:
    
    \begin{itemize}
        \item \textbf{Vysoká nevyváženost tříd:} Poměr MPP vůči PČR je cca 11:1.
        
        \item \textbf{Nekonzistentní označování:} Atribut právní kvalifikace obsahuje duplicitní významy (např. dlouhý zápis \texttt{§ 125c odst. 1...} vs. zkrácený \texttt{§ 125c/1k...}).
        
        \item \textbf{Chybějící hodnoty:} Atribut \textit{TOVZN} (Tovární značka) chybí u~12,4\,\% záznamů, atribut \textit{CASSK} (Čas) u~4,5\,\% záznamů.
        
        \item \textbf{Nestrukturovaný text:} Atribut \textit{MISTOSK} obsahuje volný text s~překlepy a nejednotným formátováním (např.: \textit{Maltézské náměstí 16, Praha 1, u domu})
    \end{itemize}

    % 5. BOD: PROFILING
    \item \textbf{Profil dat (ydata-profiling)}
    
    Detailní automaticky generovaný profil datové sady vytvořený nástrojem \texttt{ydata-profiling} je přiložen v~Příloze~A.

\end{itemize}


% --------------------------------------------------------------------

\section{Formulace cíle}
\begin{itemize}[leftmargin=*]
    \item \textbf{Přesná formulace predikčního/klasifikačního cíle.}
    
    Hlavním cílem této semestrální práce je navrhnout, natrénovat a vyhodnotit model strojového učení, který bude schopen rozlišit, kdo zaznamenal dopravní přestupek na území hlavního města Prahy. Konkrétně se jedná o rozlišení mezi Městská policie Praha a Policie České republiky.
    \item \textbf{Typ úlohy a zdůvodnění výběru metody.}

    Z hlediska strojového učení se jedná o úlohu učení s učitelem, konkrétně o binární klasifikaci.
    Pro řešení úlohy byla zvolena metoda Vícevrstvého perceptronu. Předpokládáme že vztahy mezi vstupními daty a cílovou třídou jsou komplikované ale je v nich možné najít nějaký pattern, na základě kterého se policie pohybuje. Například ulice "Evropská" může implikovat PČR v noci (měření rychlosti), ale MPP ve dne (parkování). proto jsme zvolili metodu MLP která by tyto vztahy mohla najít.
    
\end{itemize}

% --------------------------------------------------------------------

\section{Předzpracování dat}

Tato kapitola popisuje technické kroky vedoucí k transformaci surových dat do podoby vhodné pro trénování klasifikačního modelu. Proces zahrnoval čištění dat, extrakci nových příznaků (feature engineering) a finální selekci atributů na základě jejich statistické významnosti.

\begin{itemize}[leftmargin=*]
    
    % 1. BOD: VYŘEŠENÉ PROBLÉMY
    \item \textbf{Vyřešené datové problémy a popis postupů}
    
    Na základě závěrů exploratorní analýzy byly adresovány následující nedostatky v kvalitě a struktuře dat:
    \begin{itemize}
        \item \textbf{Nestrukturovaný text u lokace:} Atribut \textit{MISTOSK} původně obsahoval volný text kombinující názvy ulic, čísla popisná a další upřesnění (např. \uv{před domem č.p. 5}). Pro potřeby modelování bylo nutné tento text kategorizovat. Pomocí regulárních výrazů a vyhledávání klíčových slov (např. \uv{tunel}, \uv{náměstí}, \uv{Evropská}, \uv{spojka}) byly záznamy rozděleny do čtyř obecných tříd: \textit{náměstí, tunel, hlavní ulice} a \textit{ostatní}.
        
        \item \textbf{Nekonzistentní zápis zákonů:} Atribut \textit{ZAKON} vykazoval vysokou variabilitu v zápisu totožných přestupků (rozdílné mezery, zkratky odstavců). Byla provedena normalizace textu, kdy byly pomocí vzorů (regex) sjednoceny varianty zápisu na základní tvar paragrafu a odstavce (např. unifikace na \texttt{§125c/1k}).
        
        \item \textbf{Chybějící a odlehlé hodnoty:} U atributu tovární značky (\textit{TOVZN}) byla chybějící data (cca 12\,\%) imputována zástupnou hodnotou \texttt{UNSPECIFIED}. V případě chybějící státní příslušnosti byla použita hodnota \texttt{UNKNOWN}. Pokud u záznamu chyběl čas nebo byl v chybném formátu, byla hodina přestupku nastavena na indikátor \texttt{-1}.
    \end{itemize}

    % 2. BOD: KROKY PŘEDZPRACOVÁNÍ
    \item \textbf{Kroky předzpracování (transformace a tvorba příznaků)}
    
    Cílem transformací bylo snížit dimenzionalitu dat a zachytit obecnější trendy namísto specifických detailů, které by mohly vést k přeučení modelu.
    
    \begin{description}
        \item[Časové atributy:] 
        Ze spojitého času byla vytvořena sada kategorických příznaků s cílem zachytit cyklické chování a trendy. Konkrétně byly derivovány:
        \begin{itemize}
            \item \textbf{Měsíc} (1--12).
            \item \textbf{Hodina} (0--11), pro chybějící čas nahrazeno hodnotou \texttt{-1}.
            \item \textbf{Typ dne} (binární rozdělení na pracovní den a víkend).
            \item \textbf{Roční období} (jaro, léto, podzim, zima).
            \item \textbf{Denní doba} (ráno, odpoledne, večer, noc); v případě chybějícího času nastaveno jako \texttt{none}.
        \end{itemize}
        
        \item[Typ vozidla (\texttt{CAR\_TYPE}):] 
        Atribut značky vozidla byl diskretizován dle předem vytvořeného seznamu nejfrekventovanějších značek (např. \textit{Škoda, Volvo, Hyundai}). Značky s nižším výskytem, které nebyly součástí seznamu, byly sloučeny do kategorie \texttt{OTHER}. Záznamy bez uvedené značky byly označeny jako \texttt{UNSPECIFIED}.
    \end{description}

    % 3. BOD: ZDŮVODNĚNÍ VÝBĚRU
    \item \textbf{Zdůvodnění výběru atributů}
    
    Finální sada atributů pro trénování byla zvolena na základě analýzy korelací a iterativního testování výkonu modelu. Byla vypočtena korelační matice (viz Obrázek~\ref{fig:corr_matrix}), která vizualizuje vztahy mezi nově vytvořenými příznaky a cílovou proměnnou \textit{OZNAM}.
    
    \begin{figure}[h!]
        \centering
        \includegraphics[width=1\textwidth]{correlation_matrix.png}
        \caption{Korelační matice finálních atributů (vizualizace závislosti mezi extrahovanými příznaky)}
        \label{fig:corr_matrix}
    \end{figure}
    
    Na základě této analýzy byly pro finální trénink modelu vybrány atributy: \texttt{SEASON}, \texttt{DAY\_TIME}, \texttt{PRAGUE}, \texttt{PLACE\_TYPE}, \texttt{CAR\_TYPE}, \texttt{LAW\_CLEAN} a \texttt{IS\_FIRM}.

\end{itemize}
% --------------------------------------------------------------------
\newpage
\section{Metoda strojového učení}
    Pro řešení úlohy byla zvolena metoda Vícevrstvého perceptronu (MLP). Jedná se o dopřednou umělou neuronovou síť (Feedforward Neural Network), která se skládá z jedné vstupní vrstvy, jedné či více skrytých vrstev a jedné výstupní vrstvy.

    Základním stavebním kamenem je umělý neuron, který provádí vážený součet svých vstupů, přičítá k němu prahovou hodnotu (bias) a výsledek transformuje pomocí nelineární aktivační funkce. Díky vrstvení těchto neuronů a použití nelinearit je MLP schopen aproximovat i velmi složité rozhodovací hranice v mnohadimenzionálním prostoru příznaků, což je pro naše kategorická data klíčové.
    
    My jsme pro vytvoření modelu jsme použili knihovnu \texttt{scikit-learn} a třídu \texttt{MLPClassifier}. Na základě experimentů s počtem neuronů a skrytých vrstev byla zvolena arhitektura s dvěmi skrytými vrtsvami. První skrytá vrstva s 50 neurony a druhá s 25 neurony. Aktivační funkce v obou vrstvách je ReLu.

% --------------------------------------------------------------------

\section{Experimenty a výsledky (2–3 strany)}

    Při tréninku jsme použili data z roku 2023. Ty byli rozdělené na trénovací (80\,\% dat) a testovací (20\,\% dat). Následně jsme pak ještě natrénovaný model testovali na datech z jiných let. Jako první jsme vytvořili model s dvěmi skrytými vrstvami se 100 neuronů v první a 50 neurony v druhé vrstvě. Tento model jsme používali při úpravě dat.
    
    Při analýze dat jsme si například všimli, že v datech často chybí značka vozidla a čas. Když jsme se podívali na to, kdo udělal záznam v řádcích s chybějícími hodnotami, tak jsme zjistili, že se z většiny jednalo o PČR. Například když chyběla značka auta, tak se z 59,4 \,\% jednalo o PČR, Když si vezmeme že rozložení záznamů PČR a MPP je v datech 1:11, tak se zdá že absence údajů je vlastně dobrý ukazatel toho, zda se jedná o PČR či nikoliv. Přidali jsme sloupce \texttt{MISSING\_TIME} a \texttt{MISSING\_BRAND}. Výsledky ale byli horší než když jsme jen do sloupců \texttt{HOUR} dali hodnotu -1 a \texttt{CAR\_TYPE} dali hodnotu \texttt{UNSPECIFIED}. Řádky s chybějícím časem a značkou auta jsme i zkusili odstranit a natrénovat model na datech bez nich. Model pak na datech pro rok 2023 a 2024 měl přesnost 99.9 \,\%. Pro rok 2022 ale jen 92 \,\%.

    Následně jsme zkusili měnit počet neuronů a vrstev a rozšířili jsme testovací data ještě o roky 2021 a 2020. Zkusili jsme natrénovat model s jednou skrytou vrstvou o 50 a poté o 100 neuronech. Následně modely s dvěmi vrstvami, jeden model o 50 neuronech v první skryté vrtsvě a 25 v druhé a jeden s původními 100 neurony v první vrstvě a 50 v druhé. 
    
    Všechny modely měli přesnost 99 \,\% při určování záznamů od MPP ale při určování PČR byla přesnost u některých modelů výrazně nižší (nejhorší 61 \,\% měl model s jednou skrytou vrstvou na datech z roku 2022). Graf Přesnosti na třídě PČR vidíme na obrázku níže.
    
    \begin{figure}[h!]
        \centering
        \includegraphics[width=1\linewidth]{Code_Generated_Image.png}
        \caption{Graf přesnosti na třídě PČR}
        \label{fig:placeholder}
    \end{figure}

    nejlepších výsledků dosáhl model s 50 neurony v první a 25 neurony v druhé vrstvě.
\begin{table}[h!]
    \centering
    \caption{Přesnost modelů}
    \label{tab:precision_mode}
    % Změna definice sloupců:
    % | - svislá čára
    % >{\bfseries}c - tento sloupec (c) bude automaticky tučný
    \begin{tabular}{l|c c c c}
        \textbf{Rok} & \textbf{Model (50)} & \textbf{Model (100)} & \textbf{Model (100, 50)} & \textbf{Model (50, 25)} \\
        \hline
        2024 & 0.95 & 0.96 & 0.89 & \textbf{0.95} \\
        \hline
        2023 - trénink & 0.99 & 0.99 & 0.99 & \textbf{0.99} \\
        \hline
        2022 & 0.61 & 0.82 & 0.83 & \textbf{0.96} \\
        \hline
        2021 & 0.73 & 0.88 & 0.84 & \textbf{0.97} \\
        \hline
        2020 & 0.77 & 0.89 & 0.86 & \textbf{0.97} \\
    \end{tabular}
\end{table}

% Requires: \usepackage{multirow}
\begin{table}[h]
    \centering
    \caption{výsledky modelu(50,25)}
    \label{tab:placeholder_label}
    \begin{tabular}{lcccc}
        \textbf{Rok} & \textbf{Precision (PČR)} & \textbf{Recall (PČR)} & \textbf{F1-Score (PČR)} & \textbf{přesnost} \\
        \hline
        2024 & 0.95 & 0.94 & 0.95 & 99 \% \\
        \hline
        2023 (Trénink) & 0.99 & 0.96 & 0.98 & 100 \% \\
        \hline
        2022 & 0.96 & 0.93 & 0.95 & 99 \% \\
        \hline
        2021 & 0.97 & 0.89 & 0.93 & 98 \% \\
        \hline
        2020 & 0.97 & 0.89 & 0.93 & 99 \% \\
        
    \end{tabular}
\end{table}





% --------------------------------------------------------------------

\section{Diskuse (1 strana)}
\begin{itemize}[leftmargin=*]
    \item Zhodnocení silných a slabých stránek výsledku.
    \item Co fungovalo a co ne, a proč.
    \item Možné alternativy a budoucí vylepšení.
\end{itemize}

% --------------------------------------------------------------------

\section{Závěr (0.5–1 strana)}
\begin{itemize}[leftmargin=*]
    \item Shrnutí postupu a klíčových výsledků.
    \item Vyjádření, zda byl cíl splněn a proč.
    \item Krátký přehled naučených poznatků.
\end{itemize}

% --------------------------------------------------------------------
\newpage

\section*{Prohlášení o využití generativní AI (pokud relevantní)}
Tato sekce je povinná pouze v případě, že byla při zpracování projektu využita generativní umělá inteligence.
\
\textbf{Vzorové prohlášení:}
\
\begin{quote}
V rámci vypracování této semestrální práce jsme využili nástroje generativní umělé inteligence (např. ChatGPT, Gemini, Mistral, GitHub Copilot) k následujícím účelům: \
\begin{itemize}
    \item konzultace teoretického postupu a ověření správnosti vysvětlení,
    \item návrh části kódu / kontrola kódu,
    \item jazyková korektura textu.
\end{itemize}
Veškerým použitým postupům, kódu i interpretaci výsledků rozumíme a jsme schopni je samostatně vysvětlit v rámci ústního zkoušení. Současně bereme na vědomí, že plně odpovídáme za správnost obsahu, výpočtů, kódu i závěrů uvedených v této práci.
\end{quote}

\section*{Přílohy}
\begin{itemize}[leftmargin=*]
    \item Výstupy profilování dat.
    \item Vybrané části kódu.
    \item Dodatečné grafy a tabulky.
\end{itemize}

\end{document}
